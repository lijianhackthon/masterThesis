% !Mode:: "TeX:UTF-8"


\ctitle{拉萨方言语音识别声学建模方法研究}  %封面用论文标题,自己可手动断行
\etitle{A Study on Acoustic Modeling \\for Lhasa Dialect Speech Recognition}
\caffil{天津大学计算机科学与技术学院} %学院名称
\csubjecttitle{专业}
\csubject{\uline{计算机科学与技术}}   %专业
\cauthortitle{学生姓名}     % 学位
\cauthor{\uline{李~~健}}   %学生姓名
\csupervisortitle{指导教师}
\csupervisor{\uline{李雪威~~副教授}} %导师姓名

\declaretitle{独创性声明}
\declarecontent{
本人声明所呈交的学位论文是本人在导师指导下进行的研究工作和取得的研究成果,除了文中特别加以标注和致谢之处外,论文中不包含其他人已经发表或撰写过的研究成果,也不包含为获得 {\underline{\kai\textbf{~天津大学~}}}或其他教育机构的学位或证书而使用过的材料。与我一同工作的同志对本研究所做的任何贡献均已在论文中作了明确的说明并表示了谢意。
}
\authorizationtitle{学位论文版权使用授权书}
\authorizationcontent{
本学位论文作者完全了解{\underline{\kai\textbf{~天津大学~}}}有关保留、使用学位论文的规定。特授权{\underline{\kai\textbf{~天津大学~}}} 可以将学位论文的全部或部分内容编入有关数据库进行检索,并采用影印、缩印或扫描等复制手段保存、汇编以供查阅和借阅。同意学校向国家有关部门或机构送交论文的复印件和磁盘。
}
\authorizationadd{(保密的学位论文在解密后适用本授权说明)}
\authorsigncap{学位论文作者签名:}
\supervisorsigncap{导师签名:}
\signdatecap{签字日期:}


\cdate{\CJKdigits{\the\year}年\CJKnumber{\the\month}月}
% 如需改成二零一二年四月二十五日的格式,可以直接输入,即如下所示
% \cdate{二零一二年四月二十五日}
% \cdate{\the\year 年\the\month 月 \the\day 日} % 此日期显示格式为阿拉伯数字 如2012年4月25日
\cabstract{
拉萨方言作为有调语言,声调对于区分同音字起到了关键的作用。但目前对于拉萨方言具体有几个调还存在争议,这对使用声调信息造成了困难。本研究通过调研相关文献,并结合已录制的拉萨方言音频数据库,采用四个声调的声调系统,建立了拉萨方言带调音素集合。在特征层面,尝试了使用两种不同的基频提取方法提取每一帧的基频值,再结合MFCC参数,构成声调相关的声学特征用来训练声学模型。为了验证声调信息有助于提升拉萨方言的识别结果,本研究搭建了完整的识别系统,使用不同的音素集合和输入特征,训练DNN-HMM模型,得到字级别的识别结果。实验的训练数据31.9小时,测试集2.41小时。识别结果表明,对于DNN模型,使用区分声调的音素集合,配合声调相关的输入特征,字级别错误率与基准系统相比相对下降11.1\%。该研究验证了声调信息对拉萨方言语音识别的重要性。
}

\ckeywords{语音识别~~拉萨方言~~声学模型~~声调}

\eabstract{
As a tonal language, tone of Lhasa dialect is essential to help discriminate homophones. However, it remains controversial how many tones does Lhasa dialect have. This uncertainty brings difficulty to utilize tonal information in ASR of Lhasa dialect. In this study, we created a small-scale corpus and built the ASR system for Lhasa dialect from scratch. We adopted a four-tone pattern and designed a phone set based on the four contour contrasts scheme. After comparing different acoustic modeling methods, we use DNN model to investigate influences of the tonal information on the performance of Lhasa Tibetan speech recognition. The experimental results showed that the character error rate (CER) was improved 11\% after applying the tone based phone set and pitch-related features to DNN-HMM based speech recognition by comparing to that without tonal information. This preliminary study revealed that the tonal information plays an important role in speech recognition of Tibetan Lhasa dialect.

}

\ekeywords{ASR, Lhasa dialect, Acoustic model, tone}

\makecover

\clearpage
