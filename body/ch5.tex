% !mode:: "tex:utf-8"

\chapter{总结与展望}
语音识别技术发展至今,在部分应用场景下已达到实用阶段。尤其是对于类似英语、汉语等已经被语音识别领域深入研究的语言,语音识别系统的性能已经可以和人耳媲美。但对于很多语料资源匮乏的语种来说,由于其训练数据获取困难,加上缺少语言的语音学和语言学的专业知识,对于这些语种的语音识别研究已远远落后于目前的流行技术。

拉萨方言作为少数民族语言,同样属于语料匮乏语言,想训练其语音识别系统,难点在于收集足够多的带文本标签的语音数据,同时也缺少足够的藏语语音学和语言学的相关知识。这些限制条件使得拉萨方言语音识别的研究进展缓慢,远远落后于目前英语和汉语的语音识别技术水平。在本研究从头开始搭建了拉萨方言离线识别系统,在语音数据不充足的条件下,调查了拉萨方言声调信息对语音识别性能的影响。与此同时,拉萨方言作为有调语言,声调对于区分同音字起到了关键的作用。但目前对于拉萨方言具体有几个调还存在争议,这对使用声调信息造成了困难。本研究通过调研相关文献,并结合已录制的拉萨方言音频数据库,采用四个声调的声调系统,最终建立了拉萨方言带调音素集合。在特征层面,尝试了使用两种不同的基频提取方法提取每一帧的基频值,再结合MFCC参数,构成声调相关的声学特征用来训练声学模型。为了验证声调信息有助于提升拉萨方言的识别结果,本研究搭建了完整的识别系统,使用不同的音素集合和输入特征,分别训练三音素模型和DNN-HMM模型,得到字级别的识别结果。实验的训练数据31.9小时,测试集2.41小时。识别结果表明,对于DNN模型,使用区分声调的音素集合加上MFCC特征,字级别错误率相对下降6.1\%;使用区分声调的音素集合,配合声调相关的输入特征,字级别错误率与基准系统相比相对下降11.1\%。该研究验证了声调信息对拉萨方言语音识别的重要性。

对于拉萨方言的声调类型,目前采用的是四个声调,但这是基于藏语言学家的先验知识得到的。后续可能通过分析声调的功能负载,去确定拉萨方言究竟该采用多少个声调比较合适。并且,目前的DNN-HMM声学模型已逐步被LSTM方法取代,所以后续的识别系统可以尝试使用基于LSTM的声学模型。
