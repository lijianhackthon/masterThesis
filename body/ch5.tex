% !mode:: "tex:utf-8"

\chapter{总结与展望}
语音识别技术发展至今,在部分应用场景下已达到实用阶段。尤其是对于类似英语、汉语等已经被语音识别领域深入研究的语言,语音识别系统的性能已经可以和人耳媲美。但对于很多语料资源匮乏的语种来说,由于其训练数据获取困难,加上缺少语言的语音学和语言学的专业知识,对于这些语种的语音识别研究已远远落后于目前的流行技术。

拉萨方言作为少数民族语言,同样属于语料匮乏语言,想训练其语音识别系统,难点在于收集足够多的带文本标签的语音数据,同时也缺少足够的藏语语音学和语言学的相关知识。这些限制条件使得拉萨方言语音识别的研究进展缓慢,远远落后于目前英语和汉语的语音识别技术水平。在本研究从头开始搭建了拉萨方言离线识别系统,在语音数据不充足的条件下,调查了拉萨方言声调信息对语音识别性能的影响。与此同时,拉萨方言作为有调语言,声调对于区分同音字起到了关键的作用。但目前对于拉萨方言具体有几个调还存在争议,这对使用声调信息造成了困难。本研究通过调研相关文献,并结合已录制的拉萨方言音频数据库,采用四个声调的声调系统,最终建立了拉萨方言带调音素集合。在特征层面,尝试了使用两种不同的基频提取方法提取每一帧的基频值,再结合MFCC参数,构成声调相关的声学特征用来训练声学模型。为了验证声调信息有助于提升拉萨方言的识别结果,本研究搭建了完整的识别系统,使用不同的音素集合和输入特征,分别训练DNN-HMM模型,得到字级别的识别结果。实验的训练数据31.9小时,测试集2.41小时。识别结果表明,对于DNN-HMM模型,无论是在音素集合层面还是在特征层面,加入的声调信息均能提高识别系统准确率。在使用带调音素集合的前提下,两种不同的声调提取方法给系统带来性能上的相对提升分别为11.1\% 和7.9\%,当把由不同的声调特征得到的声学模型进行融合之后,识别的准确率的相对提升为16.0\%。该研究验证了声调信息对拉萨方言语音识别的重要性。

对于拉萨方言的声调类型,目前采用的是四个声调,但这是基于藏语言学家的先验知识得到的。后续可能通过分析声调的功能负载,去确定拉萨方言究竟该采用多少个声调比较合适。并且,目前的DNN-HMM声学模型已逐步被LSTM方法取代,所以后续的识别系统可以尝试使用基于LSTM的声学模型。

另外,即使在加入了拉萨方言声调特性之后,识别结果错误率仍然很高。其主要原因集中在藏语的语言模型上,实验中使用的测试集多为拉萨口语句子,而搜集口语文本十分困难,只能使用在互联网上较容易获取的书面用语的文本,造成了模型不匹配的问题。并且,从互联网上爬取的藏语文本中存在着大量的梵文以及拼写错误,这也造成了语言模型的不准确。在后续的工作中,可以尝试使用与测试集的口语句子匹配程度较高但数据量较小的文本先训练一个较小的语言模型,再用匹配度较低但数据量较大的互联网文本数据训练一个较大的语言模型,两个模型之间做插值得到一个最佳的语言模型;同时也可以尝试使用目前流行的LSTM方法训练基于神经网络的语言模型\cite{mikolov2010recurrent},再使用N-best rescoring或lattice-rescoring来提高识别准确率\cite{mikolov2011strategies}。
