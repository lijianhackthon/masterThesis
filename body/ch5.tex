% !mode:: "tex:utf-8"

\chapter{总结与展望}
拉萨方言作为有调语言,声调对于区分同音字起到了关键的作用。但目前对于拉萨方言具体有几个调还存在争议,这对使用声调信息造成了困难。本研究通过调研相关文献,并结合已录制的拉萨方言音频数据库,采用四个声调的声调系统,包括高平调(55)、升调(13)、降调(51)、升降调(132)。最终建立了拉萨方言带调音素集合。在特征层面,尝试了使用两种不同的基频提取方法提取每一帧的基频值,再结合MFCC参数,构成声调相关的声学特征用来训练声学模型。为了验证声调信息有助于提升拉萨方言的识别结果,本研究搭建了完整的识别系统,使用不同的音素集合和输入特征,分别训练三音素模型和DNN-HMM模型,得到字级别的识别结果。实验的训练数据31.9小时,测试集2.41小时。识别结果表明,对于DNN模型,使用区分声调的音素集合加上MFCC特征,字级别错误率相对下降6.1\%;使用区分声调的音素集合,配合声调相关的输入特征,字级别错误率与基准系统相比相对下降11.1\%。该研究验证了声调信息对拉萨方言语音识别的重要性。
