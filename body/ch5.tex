% !mode:: "tex:utf-8"

\chapter{总结与展望}
拉萨方言作为有调语言,声调对于区分同音字起到了关键的作用。但目前对于拉萨方言具体有几个调还存在争议,这对使用声调信息造成了困难。本研究通过调研相关文献,并结合已录制的拉萨方言音频数据库,采用四个声调的声调系统,包括高平调(55)、升调(13)、降调(51)、升降调(132)。最终确定了拉萨方言带调音素集合,共29个声母,48个韵母,其中每个韵母用四个不同的调来区分。在特征层面,提取每一帧的基频值,再结合MFCC参数,构成声调相关的声学特征参数。为了验证声调信息有助于提升拉萨方言的识别结果,本研究搭建了完整的识别系统,使用不同的音素集合和输入特征,分别训练三音素模型和DNN-HMM模型,得到字级别的识别结果。首先使用传统的39维MFCC特征作为输入特征,音素集合使用29个声母和48个韵母;之后,输入特征不变,音素集合中的韵母使用四个声调来区分;最后,输入特征加入每一帧的基频值,音素集合中的韵母使用四个声调来区分。实验的训练数据31.9小时,测试集2.41小时。对于DNN模型,使用区分声调的音素集合,字级别错误率相对下降6.1\%,使用声调相关的输入特征,可以进一步降低识别错误率。最后,使用区分声调的音素集合,配合声调相关的输入特征,字级别错误率与基准系统相比相对下降11.1\%。该研究验证了声调信息对拉萨方言语音识别的重要性。
