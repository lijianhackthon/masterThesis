% !mode:: "tex:utf-8"

\chapter{声学模型训练}
在\ref{acoustic model introduction}节中,已经介绍了声学模型在语音识别中的作用,以及如何使用声学模型计算似然概率$P(O|\Omega)$。本章主要介绍声学模型参数估计的问题,包括传统的GMM-HMM模型和目前广泛使用的DNN-HMM模型。
\section{传统的GMM-HMM方法}
在DNN-HMM模型兴起之前,GMM-HMM作为通用算法,一直在声学模型上占主导地位,几乎所有大型的实用语音识别系统都采用GMM-HMM作为其声学模型。GMM-HMM模型使用HMM模型对语音信号的时序特性建模,使用GMM模型计算每个隐马尔科夫模型的状态发射概率。本节首先介绍GMM模型的参数估计,接着再介绍GMM-HMM模型的训练过程。
\subsection{GMM模型的定义}
GMM模型指的是混合高斯模型(Gaussian Mixture Model)。在介绍混合高斯模型之前,首先要介绍一下高斯分布。当一个连续随机变量$x$的概率密度函数满足:\begin{equation}p(x)=\frac{1}{\sqrt{2\pi\sigma}}e^{\frac{-(x-u)^2}{2\sigma^2}}\end{equation}时,随机变量$x$服从均值为$u$,方差为$\sigma^2$的高斯分布,记作$x\sim \mathcal{N}(\mu,\sigma^2)$。因为$x$为标量,因此又叫单变量的高斯分布。当$x$变为向量$X=(x_1,x_2,...,x_k)$时,$X$的高斯分布成为多变量的高斯分布,此时$X$的概率密度函数满足\begin{equation}p(X)=\frac{1}{(2\pi)^{\frac{k}{2}}|\Sigma|^{\frac{1}{2}}}e^{-\frac{1}{2}(X-\mu)^T\Sigma^{-1}(X-\mu)} \end{equation}
其中$X$和$\mu$均为k维的向量,$\Sigma$为$k\times k$的协方差矩阵。记作$X\sim \mathcal{N}(\mu,\Sigma)$。

以上是单个的多变量高斯模型,接下来是混合的多变量高斯模型。
\begin{equation}p(X)=\sum_{m=1}^M \frac{c_m}{(2\pi)^{\frac{k}{2}}|\Sigma_m|^{\frac{1}{2}}}e^{-\frac{1}{2}(X-\mu_m)^T\Sigma_m^{-1}(X-\mu_m)}\end{equation} 其中,M表示混合的高斯模型的个数,$m$表示第$m$个高斯模型,$\mu_m$表示第$m$个高斯模型的均值向量,$\Sigma_m$表示第$m$个高斯模型对应的协方差矩阵。当M足够多时,混合的多变量高斯模型可以拟合任意分布。在语音识别中,可以使用混合高斯模型去拟合39维的MFCC向量。
\subsection{GMM模型的参数估计}\label{acousticmodel}
GMM模型的参数包括混合高斯数$M$、每个高斯所占权重$c_m$、每个高斯对应的均值向量$\mu_m$和协方差矩阵$\Sigma_m$。当随机变量$X$的两个维度不相关时,其协方差为零。所以如果随机变量$X$各维度之间都不相关,此时协方差矩阵为对角阵,可以大大减少模型参数。因此,语音识别在提取MFCC特征时,需要做DCT变换将频谱变为倒谱,可以降低MFCC各维度之间的相关性,保证后续的GMM可以使用对角阵。

EM算法(Expectation-Maximization)是训练GMM模型的主要工具。EM算法为迭代算法,每次迭代都包含两步:E步和M步。其核心思想是:对于有$M$个成分的混合高斯模型,假设观测到的样本点$X$服从$M$个成分中的某一个成分$GMM_m$的概率分布,但不能确定具体是哪一个成分。如果能够得到$x$具体是由哪个成分得到的,那么整个模型就不再包含隐变量,可以直接使用最大似然估计(MLE)得到均值向量$\mu_m$ 以及方差矩阵$\Sigma_m$。EM算法不直接指定$x$具体由哪个成分得到,而是在算法的E步时,计算每个高斯成分对这个样本点的占有概率。得到这些后验概率之后,整个模型不再包含隐变量,接下来就是M步。通过最大似然估计,更新参数$c_m$、$mu_m$、$\Sigma_m$。EM算法的具体公式如下。

E步:
\begin{equation}
  h_m^{(j)}(t) = \frac{c_m^{(j)}\mathcal{N}(X^{(t)};\mu_m^{(j)},\Sigma_m^{(j)})}{\sum_{i=1}^M c_i^{(j)}\mathcal{N}(X^{(t)};\mu_i^{(j)},\Sigma_i^{(j)})}
\end{equation}

M步:
\begin{eqnarray}
  &c_m^{(j+1)}=\frac{1}{N}\sum_{t=1}^N h_m^{(j)}(t) \\
  &\mu_m^{(j+1)}=\frac{\sum_{t=1}^{N} h_m^{(j)}(t)X^{(t)}}{\sum_{t=1}^{N} h_m^{(j)}(t)} \\
  &\Sigma_m^{(j+1)}=\frac{\sum_{t=1}^{N} h_m^{(j)}(t)[X^{(t)}-\mu_m^{(j)}][X^{(t)}-\mu_m^{(j)}]^T}{\sum_{t=1}^{N} h_m^{(j)}(t)}
\end{eqnarray}
其中,M表示共有M个高斯成分、N表示样本点个数、$X^{(t)}$表示第$t$个样本点、$m$表示第$m$个高斯成分、$c_m$为第$m$个高斯所占权重、$\mu_m$为第$m$个高斯的均值向量、$\Sigma_m$为第$m$个高斯的协方差矩阵。

由于第一次迭代时,需要已知各个高斯成分的参数,因此使用EM算法时需要初始化,并且,EM算法收敛于局部最优。
\subsection{GMM-HMM模型训练}
GMM-HMM模型训练使用前后向算法,又称作Baum-Welch算法。Baum-Welch算法是EM算法在训练HMM模型上的具体应用。用前后向算法估计HMM模型状态转移概率的整体流程如下:
%\begin{enumerate}
%  \item 首先通过前向算法,计算$\alpha(S,t)$和$P(X)$。
%  \item 再通过后向算法,计算$\beta(S,t)$。
%  \item 对于每一个弧$S\rightarrow^{X_t}S^{'}$和$S \xrightarrow{X_t} S^{'}$时间步$t$,计算这个弧在时间步$t$产生的观测$X=X_t$的概率\begin{equation}c(S\rightarrow^{X_t},t)=\frac{1}{P(X)}\times P_{S\rightarrow^{X_t}S^{'}}\times \alpha(S,t-1) \times \beta(S^{'},t)\end{equation}
%  \item 对所有时间步下由状态$S$跳转到$S^{'}$的所有情况求和。\begin{equation}c(S\rightarrow S^{})=\sum_{t=1}^T c(S\rightarrow^X S^{'},t)\end{equation}
%  \item 最后,根据最大似然估计,从状态$S$到状态$S^{'}$的状态转移\begin{equation}P_{S\rightarrow^X S^{'}}^{MLE}=\frac{c(S\rightarrow^X S^{'})}{\sum_{X,S^{'}} c(S\rightarrow^X S^{'})}\end{equation}
%\end{enumerate}
\begin{enumerate}
  \item 首先通过前向算法,计算$\alpha(S,t)$和$P(X)$。
  \item 再通过后向算法,计算$\beta(S,t)$。
  \item 对于每一个弧$S\xrightarrow{X_t}S^{'}$和$S \xrightarrow{X_t} S^{'}$时间步$t$,计算这个弧在时间步$t$产生的观测$X=X_t$的概率\begin{equation}c(S\xrightarrow {X_t},t)=\frac{1}{P(X)}\times P_{S\xrightarrow {X_t}S^{'}} \times \alpha(S,t-1) \times \beta(S^{'},t)\end{equation}
  \item 对所有时间步下由状态$S$跳转到$S^{'}$的所有情况求和。\begin{equation}c(S\rightarrow S^{})=\sum_{t=1}^T c(S\xrightarrow X S^{'},t)\end{equation}
  \item 最后,根据最大似然估计,从状态$S$到状态$S^{'}$的状态转移\begin{equation}P_{S\xrightarrow X S^{'}}^{MLE} =\frac{c(S\xrightarrow X S^{'})}{\sum_{X,S^{'}} c(S\xrightarrow X S^{'})}\end{equation}
\end{enumerate}
以上为使用Baum-Welch算法计算HMM状态转移概率的过程。

尽管GMM-HMM模型被成功应用到声学建模中,但GMM-HMM框架本身依然存在很多问题,比如HMM模型本身的条件独立假设的限制。随着神经网络模型理论及深度学习技术的发展繁荣,目前DNN模型已经取代了GMM模型,并且人们也逐渐使用RNN模型(循环神经网络)替换HMM模型,来对语音信号的时序特性建模。
\section{目前广泛使用的DNN-HMM方法}
在DNN模型被广泛应用到语音识别之前,GMM-HMM一直在声学模型上占主导地位。2006年之后,Hinton等人提出了DNN预训练方法,通过预训练受限玻尔兹曼机来初始化DNN模型,使得训练深度的神经网络成为可能。之后,DNN模型逐步取代GMM模型,DNN-HMM成为声学建模的主流方法。与GMM-HMM相比,DNN-HMM的优势在于使用DNN这一强大的判别模型替代传统的GMM模型来计算HMM的状态发射概率。图~\ref{fig:dnnhmm}~为DNN-HMM声学模型的结构框图。
\begin{figure}[htbp]
\centering
\includegraphics[width=1.0\textwidth]{dnnhmm}
\caption{DNN-HMM声学模型结构图\cite{yu2012automatic}}\label{fig:dnnhmm}
\vspace{\baselineskip}
\end{figure}
\subsection{DNN-HMM模型训练步骤}
DNN-HMM模型的训练可以使用内嵌的维特比算法(Embedded Viterbi)实现。DNN-HMM模型总共包含三个部分:DNN模型、HMM模型、状态的先验概率分布。由于DNN-HMM与GMM-HMM使用同样的三音素模型和HMM模型,所以在训练DNN-HMM之前,首先要训练GMM-HMM模型。由于DNN-HMM在模型训练时需要的每个句子的音素标注是由GMM-HMM模型对每个句子做强制对齐时得到的,因此得到一个好的GMM-HMM模型对训练DNN-HMM模型十分重要。训练好GMM-HMM模型之后,使用GMM-HMM模型对训练集的每个句子使用维特比算法得到状态对齐信息。再由状态对齐信息通过映射关系得到每个状态对应的senone。再通过切分信息将senone映射到实际的输入特征上。同时通过senone到特征的映射得到每个状态的先验概率分布。复制已经得到的GMM-HMM模型中的HMM模型,用作DNN-HMM模型中的HMM。接下来是DNN模型的预训练,通过训练深度置信网络(DBN),将DBN的模型参数作为DNN的初始化参数,接着使用反向传播(Back Propagation)算法进行随机梯度下降,最终得到DNN模型。至此,DNN-HMM模型参数已全部得到。
