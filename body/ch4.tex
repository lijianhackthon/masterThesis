% !mode:: "tex:utf-8"

\chapter{拉萨方言语音识别}
拉萨方言作为语料匮乏语言的一种,想训练其语音识别系统,难点在于收集足够多的带文本标签的语音数据,同时也缺少足够的藏语语音学和语言学的相关知识。这些限制条件使得拉萨方言语音识别的研究进展缓慢,远远落后于目前英语和汉语的语音识别技术水平。在本研究中,我们录制了一个小规模的拉萨方言数据库,从头开始搭建了拉萨方言离线识别系统。在语音数据不充足的条件下,调查了拉萨方言声调信息对语音识别性能的影响。由于拉萨方言目前还没有一个公认的声调系统,因此本论文中,我们采用了四个声调的声调模式,设计了带调的音素集合,并且在特征层面加入了声调相关的信息。实验结果表明,利用声调信息之后,识别的准确率相对提升16.0\%。以下是拉萨方言语音识别系统实验的详细介绍。
\section{拉萨方言语音识别研究现状}
拉萨方言属于西藏中部方言的一种,使用者包括拉萨以及周边地区的居民。由于拉萨在政治和文化方面的重要性,拉萨方言相关的研究近些年开始受到越来越多的关注。现有的拉萨方言相关研究工作,其中一部分为拉萨方言声学模型的研究。[2]是较早的使用DTW做拉萨方言孤立词识别的,[3,4]是使用传统GMM做拉萨方言连续语音识别的。很明显,以上的研究所用技术已经落后于目前流行的方法。[5,6]关注的是如何使用神经网络的共享隐层解决拉萨方言训练数据不足的问题。虽然[5,6]等最新的研究使用了流行的深度学习技术,但是很少有相关工作提到如何使用拉萨方言本身的特性去提高识别性能。

拉萨方言属于单音节的有调语言,每个拉萨方言的单字都是一个带调的音节,声调在区分同音字上扮演着很重要的角色,尤其是在缺少较强的上下文信息的情况下。然而,很少有研究提到如何利用拉萨方言的声调去提高系统的识别性能。如果能对拉萨方言的声调建立准确模型,将会在很大程度上提高识别系统的准确度。
\section{拉萨方言数据库及发音字典}
藏语拉萨方言数据库由天津大学认知计算与应用重点实验室(CCA)与中国社会科学院民族学与人类学研究所合作录制,共包含13名男性发音人和10名女性发音人。发音人均是以拉萨方言为母语的中央民族大学本科生。每位发音人录制相同的3,100句藏语音素平衡句,句子的平均时长为3.2秒。录制环境为安静的办公室环境。音频信号的获取采用单声道、16KHz采样率、16bit量化,保存为.wav格式的音频文件。数据经过人工校对,剔除掉不合格的音频数据,最后可用的数据总时长为35.92小时。将数据库分为三个部分,其中训练集数据用作模型训练,包含7名女性发音人和10名男性发音人,共36,090个句子,总时长为31.9小时;测试集用作模型测试,包含3名女性发音人和3名男性发音人,共2,664个句子,总时长为2.41小时;开发集用于选择模型参数,其发音人和训练集发音人一致,共1700个句子,总时长为1.51小时。训练集和测试集发音人和发音句子没有交叉,保证了模型测试结果的准确性。

发音字典由合作单位中国社会科学院民族学与人类学研究所提供,字典采用声韵母组合的规则,共包含29个声母,48个韵母。字典条目为2,100。基本涵盖了所有藏语拉萨方言数据库中出现的藏文字。下表是发音字典所用声韵母集合:{\color{red}{添加拉萨声韵母集合}}
%\section{语言模型训练}
%语言模型的训练数据包含两个部分,一部分是爬取的维基百科上的藏语文本数据;另外一部分是藏族五省区中学课本的一部分文本。总共14,430个藏语句子。
\section{拉萨方言语音识别基准系统}
为了验证声调特征的有效性,我们首先搭建了基准系统。

\subsection{声调相关特征}
\subsection{模型融合}
\section{识别结果及分析}
