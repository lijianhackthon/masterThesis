% !mode:: "tex:utf-8"
\chapter{绪论}

\section{自动语音识别简介}
语音交互是人类社会最直接、最自然的沟通交流方式,而机器作为辅助人类生产及日常生活的工具,目前人类与各种机器交互的方式更多的还是依赖于键盘、鼠标、显示器等输入输出设备。如何摆脱鼠标、键盘,使得人与机器之间的沟通像人与人之间的沟通那样自然,是智能时代人类面临的重大挑战。想要实现人机对话,需要涉及语音识别、自然语言理解、语音合成等关键技术。其中,语音识别作为关键部分之一,就像机器的耳朵一样,机器需要依靠它来辨别人类到底在说些什么。

近些年来,随着科技的发展,在安静环境下使用近距离麦克风的应用场合,语音识别已达到实用阶段。纵观语音识别的发展史,20世纪50年代,贝尔实验室成功研制出世界上第一个语音识别系统Audrey,方法是基于元音的共振峰的测量,虽然该系统为针对特定说话人的孤立词识别,且只能识别十个英文数字的发音,但这意味着语音识别的时代开启了。20世纪60年代至70年代之间,语音识别领域取得了突破性进展。线性预测编码(Linear Predictive Coding)被应用在声学特征的提取上;动态时间规整(Dynamic Time Warping)技术用来解决模板匹配时非线性时间对齐的问题。这些关键性的突破使得特定说话人的孤立词识别成为可能。20世纪80年代,语音识别的任务开始从孤立词识别转向连续语音识别,比如识别连续朗读的数字串等。这一时期的重大进展是语音识别方法从模板匹配转为基于统计模型方法,其中最突出的是隐马尔科夫模型(Hidden Markov Model),该模型基于马尔科夫假设,实现了对时间序列结构的建模。该方法从80年代中期开始逐渐被世界范围内的研究机构广泛接受并成为主流的语音识别方法,直到今天,很多成熟的大规模连续语音识别系统依然没有脱离HMM的方法框架。20世纪90年代出现了很多判别训练方法,包括最小识别误差MCE(Minimum Classification Error)和最大互信息MMI(Maximum Mutual Information)等。相比于最大似然估计的训练方法,这些判别训练方法能够提供更好的识别性能。{\color{red}{\large 根据哥大ASR Lecture1 37页补充内容}}自2006年Hinton等人提出有效的训练深度神经网络算法开始,深度学习技术逐渐流行并在多个领域取得显著成果。在语音识别领域,深度学习用来进行声学模型建模并获得巨大成功,尤其是对于大规模的识别任务而言。这要得益于反向传播算法的使用,以及越来越多的计算资源和训练数据。
\section{本文研究内容及各章节安排}
自20世纪60年代开始,近60年的技术积累使得语音识别性能已达到实用阶段,在某些特定的语音识别任务上,机器甚至已经超过人类。尤其是近几年深度神经网络取代传统的GMM(Gaussian Mixture Model)模型,使得识别率得到历史性突破。然而,这些性能上的突破大多都是针对英语、普通话等语料充足且已经被广泛研究理解的语种。对于许多语料匮乏语种来说,语言识别还停留在很初级的阶段。比如以藏语拉萨方言为例,目前还没有公开的比较成熟的语料库,同时也缺乏相应的拉萨方言的语音学知识,且由于藏语本身语言特性复杂,训练一个实用的语言模型十分困难,这些问题导致现阶段几乎还没有实用的藏语大规模连续语音识别系统。目前现有的关于藏语语音识别的研究主要集中在特征提取以及使用动态贝叶斯网络构建声学模型上。应用深度学习技术来对藏语声学模型建模的研究还非常少,藏语识别的研究也处在刚刚起步的阶段。即使是用于训练藏语声学模型的音素集合都还没有一个统一的参考标准。本研究从录制拉萨方言平衡语料库开始,设计了拉萨方言发音字典,尝试使用GMM-HMM、DNN-HMM及Tandem等方法训练了声学模型,通过爬取网络上的藏语文本数据训练得到语言模型,搭建了离线的拉萨方言语音识别系统。并且首次探索了如何利用拉萨方言的声调信息提高识别准确率。

本论文的章节安排如下:第一张为绪论部分,简要介绍了语音识别的任务及发展史;第二章为背景介绍,主要讲述了语音识别系统的各个组成部分及评价指标,并且介绍了本工作涉及到的语音识别工具箱。第三章总结了训练声学模型的各种方法,包括传统的GMM-HMM、目前广泛使用的DNN-HMM方法以及逐渐兴起的LSTM方法;第四章详细描述了搭建拉萨方言语音识别系统的过程,包括声学模型和语言模型的训练,以及声调特征提取的相关实验;第五章为总结和展望。
